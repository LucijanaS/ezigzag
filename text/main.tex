\documentclass[12pt]{article}

\usepackage{hyperref}
\usepackage{natbib}
\usepackage{amsmath}

\title{}

\author{The usual suspects}

\date{January 2025}

\begin{document}

\maketitle

\section{Ray-tracing with multiple lens planes}

Let us adopt the following notation.
\begin{itemize}
\item We associate subscript 0 with the observer, 1 with the
  lower-redshift lens, 2~with the other lens, and 3~with the source.
\item We write $D_{01}$ and so on for the various angular-diameter
  distances.
\item We write $x_1$ and so on for the physical transverse location of
  a light ray.
\item We let $\theta$ be the observed angular position and write
  $\hat\alpha(x_1)$ for lensing deflection angles.
\end{itemize}
The multi-plane lensing equations are then as follows.\footnote{These
are equivalent to the equations in \cite{2016MNRAS.456.2210C}. The
earlier incorrect thinking effectively repeated $D_{02}$ and $D_{12}$
in the third line, rather than replacing them with $D_{03}$ and
$D_{13}$.  Fortunately in this lens the redshifts are such that the
differences are fairly small.}
\begin{equation}
\begin{aligned}
  x_1 &= D_{01} \, \theta \\
  x_2 &= D_{02} \, \theta - D_{12} \, \hat\alpha(x_1) \\
  x_3 &= D_{03} \, \theta - D_{13} \, \hat\alpha(x_1)
                          - D_{23} \, \hat\alpha(x_2)
\end{aligned}
\end{equation}
Here each line traces a ray backwards from the observer.



The lens J1721+8842 has $z_1=0.184$, $z2=1.885$, $z_3=2.38$
\citep{2018MNRAS.479.5060L,2024arXiv241104177D}.


\newcommand{\mnras}{MNRAS}

\bibliographystyle{aa}
\bibliography{main}

\end{document}
