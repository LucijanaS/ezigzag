\documentclass[12pt]{article}

\usepackage{hyperref}
\usepackage{natbib}
\usepackage{amsmath}

\title{}

\author{The usual suspects}

\date{January 2025}

\begin{document}

\maketitle

\section{Introduction}

\begin{equation}
\begin{aligned}
  x_1 &= D_{01} \theta \\
  x_2 &= D_{02} \theta - D_{12} \alpha(x_1) \\
  x_3 &= D_{03} \theta - D_{13} \alpha(x_1) - D_{23} \alpha(x_2)
\end{aligned}
\end{equation}
Checked against \cite{2016MNRAS.456.2210C}.\footnote{The earlier
incorrect thinking effectively repeated $D_{02}$ and $D_{12}$ in the
third line, rather than replacing them with $D_{03}$ and $D_{13}$.
Fortunately in this lens the redshifts are such that the differences
are fairly small.}

The lens J1721+8842 has \citep{2018MNRAS.479.5060L,2024arXiv241104177D}
$z_1=0.184$, $z2=1.885$, $z_3=2.38$.

\newcommand{\mnras}{MNRAS}

\bibliographystyle{aa}
\bibliography{main}

\end{document}
